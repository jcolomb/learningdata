\documentclass[]{article}
\usepackage{lmodern}
\usepackage{amssymb,amsmath}
\usepackage{ifxetex,ifluatex}
\usepackage{fixltx2e} % provides \textsubscript
\ifnum 0\ifxetex 1\fi\ifluatex 1\fi=0 % if pdftex
  \usepackage[T1]{fontenc}
  \usepackage[utf8]{inputenc}
\else % if luatex or xelatex
  \ifxetex
    \usepackage{mathspec}
    \usepackage{xltxtra,xunicode}
  \else
    \usepackage{fontspec}
  \fi
  \defaultfontfeatures{Mapping=tex-text,Scale=MatchLowercase}
  \newcommand{\euro}{€}
\fi
% use upquote if available, for straight quotes in verbatim environments
\IfFileExists{upquote.sty}{\usepackage{upquote}}{}
% use microtype if available
\IfFileExists{microtype.sty}{%
\usepackage{microtype}
\UseMicrotypeSet[protrusion]{basicmath} % disable protrusion for tt fonts
}{}
\usepackage[margin=1in]{geometry}
\usepackage{graphicx}
\makeatletter
\def\maxwidth{\ifdim\Gin@nat@width>\linewidth\linewidth\else\Gin@nat@width\fi}
\def\maxheight{\ifdim\Gin@nat@height>\textheight\textheight\else\Gin@nat@height\fi}
\makeatother
% Scale images if necessary, so that they will not overflow the page
% margins by default, and it is still possible to overwrite the defaults
% using explicit options in \includegraphics[width, height, ...]{}
\setkeys{Gin}{width=\maxwidth,height=\maxheight,keepaspectratio}
\ifxetex
  \usepackage[setpagesize=false, % page size defined by xetex
              unicode=false, % unicode breaks when used with xetex
              xetex]{hyperref}
\else
  \usepackage[unicode=true]{hyperref}
\fi
\hypersetup{breaklinks=true,
            bookmarks=true,
            pdfauthor={julien},
            pdftitle={figuremaking},
            colorlinks=true,
            citecolor=blue,
            urlcolor=blue,
            linkcolor=magenta,
            pdfborder={0 0 0}}
\urlstyle{same}  % don't use monospace font for urls
\setlength{\parindent}{0pt}
\setlength{\parskip}{6pt plus 2pt minus 1pt}
\setlength{\emergencystretch}{3em}  % prevent overfull lines
\setcounter{secnumdepth}{0}

%%% Use protect on footnotes to avoid problems with footnotes in titles
\let\rmarkdownfootnote\footnote%
\def\footnote{\protect\rmarkdownfootnote}

%%% Change title format to be more compact
\usepackage{titling}

% Create subtitle command for use in maketitle
\newcommand{\subtitle}[1]{
  \posttitle{
    \begin{center}\large#1\end{center}
    }
}

\setlength{\droptitle}{-2em}
  \title{figuremaking}
  \pretitle{\vspace{\droptitle}\centering\huge}
  \posttitle{\par}
  \author{julien}
  \preauthor{\centering\large\emph}
  \postauthor{\par}
  \predate{\centering\large\emph}
  \postdate{\par}
  \date{25 Sep 2015}



\begin{document}

\maketitle


\begin{figure}[htbp]
\centering
\includegraphics{firsttest_files/figure-latex/unnamed-chunk-2-1.pdf}
\caption{\label{fig:PKC} Our attempt at determining the PKC gene
involved has failed. Performance indexes (PI) during a test period
following an 8 min. training session is reported. LEFT: Flies putatively
mutants for PKC genes (PKC-53e, PKC-delta and PCK-InaC) performed well
in the self-learning assay. Right: Flies possessing RNAi constructs
designed against PKC53e and PKC InaC were crossed to
elav-Gal4;tub-Gal80ts or to CS females. RNAi was induced for two days
before the experiment via a 32 degrees heat shock. While the construct
against PKC InaC had no effect, the construct for PKC53e prevented
self-learning formation even in absence of Gal4 driven expression, such
that no firm conclusion can be driven. Full genotype of the flies tested
is indicated below. CS x 53eRi\_V : ;;UAS\_PKC53eRNAi\_27696/+ .
elavG4;tG80 x PKC53eRi\_V :
elavGal4/+;tubGal80ts/+;UAS\_PKC53eRNAI\_27696/+ . elavG4;tG80 x
PKCInacRi : elavGal4/+;tubGal80ts/+;UAS\_PKCInacRNAI\_2895/+ . Stars
indicate significant difference of the score against 0, using a
non-parametrical wilcox test.}
\end{figure}

\begin{figure}[htbp]
\centering
\includegraphics{firsttest_files/figure-latex/unnamed-chunk-3-1.pdf}
\caption{\label{fig:heatshock} PKC inhibition (achieved by using an
effective heat shock protocol) in neurons, but not central brain
regions, prevents self-learning formation. Driving Gal4 in all neurons
using the elav-Gal4 driver while inactivating its ubiquitously expressed
inhibitor Gal80 with a heat shock protocol can drive the expression of
the PKC inhibitor. While test flies are still performing well after a
mild heat shock (A, data pooled from different protocols: 33° for 15h,
36° for 2h, and 37°for 1h), astrong heat shock prevent learning in
control flies (B, 37° for 2h). After a 4 hours heat shock at 35°C, test
but not control flies were unable to form self-learning (C). Using this
latter protocol, we restricted the expression of Gal4 in central brain
regions using different drivers targetting central brain regions (D),
which were all ineffective in preventing self-learning. Full genotype of
the flies tested is indicated below. elavG4 xCS : elav-Gal4/+ . elavG4 x
T\_PKCi : elav-Gal4/+;tubGal80ts/+ ; UAS-PKCi/+ . CS x T\_PKCi :
tubGal80ts/+ ; UAS-PKCi/+. 7y;c819G4 x T\_PKCi : tubGal80ts/+ ;
UAS-PKCi/+ \_\_ H24-Gal4 . c232G4 x T\_PKCi : tubGal80ts/+ ;
UAS-PKCi/7y\_Gal4,c819-Gal4 . H24G4 x T\_PKCi : tubGal80ts/+ ;
UAS-PKCi/c232-Gal4 . Stars indicate significant difference of the score
against 0, using a non-parametrical wilcox test.}
\end{figure}

\begin{figure}[htbp]
\centering
\includegraphics{firsttest_files/figure-latex/unnamed-chunk-4-1.pdf}
\caption{\label{fig:motoneurons} Flies with PKC inhibition using Gal4
line showing expression in motoneurons, were impaired in self-learning.
A. Using OK371-Gal4 (expression in most glutamatergic neurons) or
d42-Gal4 to drive PKC inhibition was effective in preventing
self-learning formation, while the control flies seem to learn, although
the score of the d42Gal4 x CS control was not statistically
significantly positive. B. While the previous result with the D42Gal4
driver was reproduced, c380-Gal4, a second driver showing expression in
motoneurons, lead to similar effects. C. The use of the d42Gal4,chaGal80
double construct as a driver was effective in preventing self-learning,
while the controls did perform well Heat shock protocol was a 4 hours
heat shock at 35°C. Full genotype of the flies tested is indicated
below. CS x T\_PKCi : tubGal80ts/+ ; UAS-PKCi/+ . d42G4 x T\_PKCi :
tubGal80ts/+ ; UAS-PKCi/d42-Gal4 . OK371G4 x T\_PKCi : tubGal80ts/+ ;
UAS-PKCi/+;OK371/+ . d42G4 x CS : d42Gal4/+ . OK371 x CS : OK371/+
.c380G4 x T\_PKCi : c380-Gal4/+ . c380G4 x CS : c380Gal4/+; tubGal80ts/+
; UAS-PKCi/+ . d42G4,chaG80 x CS : d42-Gal4, cha-Gal80/+ . d42G4,chaG80
x T\_PKCi : tubGal80ts/+ ; UAS-PKCi/d42-Gal4, cha-Gal80 . CS x T\_PKCi :
tubGal80ts/+ ; UAS-PKCi/+ . Stars indicate significant difference of the
score against 0, using a non-parametrical wilcox test.}
\end{figure}

\end{document}
